%%%%%%%%%%%%%%%%%%%%%%%%%%%%%%%%%%%%%%%%%%%%%%%%%%%%%%%%%%%%%%%%%%%%%%%%%%%%
% AGUJournalTemplate.tex: this template file is for articles formatted with LaTeX
%
% This file includes commands and instructions
% given in the order necessary to produce a final output that will
% satisfy AGU requirements, including customized APA reference formatting.
%
% You may copy this file and give it your
% article name, and enter your text.
%
%
% Step 1: Set the \documentclass
%
%

%% To submit your paper:
\documentclass[draft]{agujournal2019}
\usepackage{url} %this package should fix any errors with URLs in refs.
\usepackage{lineno}
\usepackage[inline]{trackchanges} %for better track changes. finalnew option will compile document with changes incorporated.
\usepackage{soul}
\linenumbers
%%%%%%%
% As of 2018 we recommend use of the TrackChanges package to mark revisions.
% The trackchanges package adds five new LaTeX commands:
%
%  \note[editor]{The note}
%  \annote[editor]{Text to annotate}{The note}
%  \add[editor]{Text to add}
%  \remove[editor]{Text to remove}
%  \change[editor]{Text to remove}{Text to add}
%
% complete documentation is here: http://trackchanges.sourceforge.net/
%%%%%%%

%\draftfalse

%% Enter journal name below.
%% Choose from this list of Journals:
%
% JGR: Atmospheres
% JGR: Biogeosciences
% JGR: Earth Surface
% JGR: Oceans
% JGR: Planets
% JGR: Solid Earth
% JGR: Space Physics
% Global Biogeochemical Cycles
% Geophysical Research Letters
% Paleoceanography and Paleoclimatology
% Radio Science
% Reviews of Geophysics
% Tectonics
% Space Weather
% Water Resources Research
% Geochemistry, Geophysics, Geosystems
% Journal of Advances in Modeling Earth Systems (JAMES)
% Earth's Future
% Earth and Space Science
% Geohealth
%
\journalname{Geophysical Research Letters}

\begin{document}

%% ------------------------------------------------------------------------ %%
%  Title
%
% (A title should be specific, informative, and brief. Use
% abbreviations only if they are defined in the abstract. Titles that
% start with general keywords then specific terms are optimized in
% searches)
%
%% ------------------------------------------------------------------------ %%



\title{Surface Winds and Enthalpy Fluxes During Tropical Cyclone Formation From Easterly Waves: A CYGNSS view}


%% ------------------------------------------------------------------------ %%
%
%  AUTHORS AND AFFILIATIONS
%
%% ------------------------------------------------------------------------ %%

% Authors are individuals who have significantly contributed to the
% research and preparation of the article. Group authors are allowed, if
% each author in the group is separately identified in an appendix.)

% List authors by first name or initial followed by last name and
% separated by commas. Use \affil{} to number affiliations, and
% \thanks{} for author notes.
% Additional author notes should be indicated with \thanks{} (for
% example, for current addresses).

% Example: \authors{A. B. Author\affil{1}\thanks{Current address, Antartica}, B. C. Author\affil{2,3}, and D. E.
% Author\affil{3,4}\thanks{Also funded by Monsanto.}}


\authors{Anantha Aiyyer\affil{1}  and Carl Schreck\affil{2}}

\affiliation{1}{Department of Marine Earth and Atmospheric Sciences, North Carolina State University, Raleigh, NC USA.}


\affiliation{2}{Cooperative Institute for Satellite Earth System Studies (CISESS), North Carolina State University, Asheville, NC}

% \affiliation{1}{First Affiliation}
% \affiliation{2}{Second Affiliation}
% \affiliation{3}{Third Affiliation}
% \affiliation{4}{Fourth Affiliation}

%(repeat as many times as is necessary)

%% Corresponding Author:
% Corresponding author mailing address and e-mail address:

% (include name and email addresses of the corresponding author.  More
% than one corresponding author is allowed in this LaTeX file and for
% publication; but only one corresponding author is allowed in our
% editorial system.)

% Example: \correspondingauthor{First and Last Name}{email@address.edu}

\correspondingauthor{Anantha Aiyyer}{aaiyyer@ncsu.edu}

%% Keypoints, final entry on title page.

%  List up to three key points (at least one is required)
%  Key Points summarize the main points and conclusions of the article
%  Each must be 100 characters or less with no special characters or punctuation and must be complete sentences

% Example:
% \begin{keypoints}
% \item	List up to three key points (at least one is required)
% \item	Key Points summarize the main points and conclusions of the article
% \item	Each must be 100 characters or less with no special characters or punctuation and must be complete sentences
% \end{keypoints}

\begin{keypoints}
\item CYGNSS winds clearly depict the climatological easterly wave stormtrack

\item  A precursor vortex (proto-vortex) is seen in surface wind fields as early as 3 days prior to the tropical cyclone formation.

\item An inward increase of surface enthalpy fluxes within the proto-vortex is seen prior to the tropical cyclone formation. 


%This is consistent with a recent modeling study that concluded that such a radial structure of surface enthalpy (i.e., heat) fluxes is important for tropical cyclone formation.

\end{keypoints}

%% ------------------------------------------------------------------------ %%
%
%  ABSTRACT and PLAIN LANGUAGE SUMMARY
%
% A good Abstract will begin with a short description of the problem
% being addressed, briefly describe the new data or analyses, then
% briefly states the main conclusion(s) and how they are supported and
% uncertainties.

% The Plain Language Summary should be written for a broad audience,
% including journalists and the science-interested public, that will not have 
% a background in your field.
%
% A Plain Language Summary is required in GRL, JGR: Planets, JGR: Biogeosciences,
% JGR: Oceans, G-Cubed, Reviews of Geophysics, and JAMES.
% see http://sharingscience.agu.org/creating-plain-language-summary/)
%
%% ------------------------------------------------------------------------ %%

%% \begin{abstract} starts the second page

\begin{abstract}
We examined the Cyclone Global Navigation Satellite System (CYGNSS) retrievals of surface winds and enthalpy fluxes in African easterly waves that led to the formation of 31 Atlantic tropical cyclones from 2018--2021. Lag composites show a cyclonic proto-vortex as early as 3 days prior to tropical cyclogenesis. The distribution of enthalpy fluxes within the proto-vortex does not vary substantially prior to cyclogenesis, but subsequently, there is an increase in the upper extreme values. A negative radial gradient of enthalpy fluxes becomes apparent as early as 2 days before cyclogenesis. These results---based on a novel data blending satellite retrievals and global reanalysis---are consistent with recent studies that have found that tropical cyclone spin-up is associated with a shift of peak convection towards the vortex-core and a radially inward increase of enthalpy fluxes. They provide additional evidence for the importance of surface enthalpy fluxes and their radial structure for tropical cyclogenesis.

%We examined the Cyclone Global Navigation Satellite System (CYGNSS) retrievals of surface winds and enthalpy (latent and sensible heat)  fluxes in African easterly waves (AEWs) that led to the formation of 31 Atlantic tropical cyclones from 2018--2021.  Lag composites of CYGNSS wind speeds, along with 10 m wind vectors from a global reanalysis, show a cyclonic proto-vortex as early as 3 days prior to tropical cyclogenesis. The distribution of enthalpy fluxes within a region bounding the proto-vortex does not vary substantially prior to tropical cyclogenesis. Subsequently, the upper extremes ($>90^{th}$ percentile values) show robust increases. Importantly, a negative radial gradient of surface enthalpy fluxes becomes apparent as early as 2 days prior to cyclogenesis. These results are consistent with recent observational and modeling studies that have found that tropical cyclone spin-up is associated with a shift of peak convection towards the vortex-core and a radially inward increase of surface enthalpy fluxes. Our results---based on a novel data set that blends satellite wind retrievals and global reanalysis---provide additional evidence for the importance of surface enthalpy fluxes and their radial structure for tropical cyclogenesis.
\end{abstract}


\section*{Plain Language Summary}
We used data derived from the recently launched Cyclone Global Navigation Satellite System (CYGNSS) to examine the surface winds and heat fluxes during the transition of easterly waves to tropical cyclones in the Atlantic. The CYGNSS winds show a proto-vortex in place 3 days before the formation of the tropical cyclone. The heat fluxes from the ocean to the air---which fuel the tropical cyclone---are enhanced near the core of the developing vortex as compared to the outer regions. This is consistent with past theoretical and observational studies, and likely contributes to the development of a deep moist column of air that typically precedes tropical cyclogenesis. The novelty of this paper lies in the use of a new data set and emphasis on the period leading up to tropical cyclogenesis from easterly waves.



%% ------------------------------------------------------------------------ %%
%
%  TEXT
%
%% ------------------------------------------------------------------------ %%

%%% Suggested section heads:
% \section{Introduction}
%
% The main text should start with an introduction. Except for short
% manuscripts (such as comments and replies), the text should be divided
% into sections, each with its own heading.

% Headings should be sentence fragments and do not begin with a
% lowercase letter or number. Examples of good headings are:

% \section{Materials and Methods}
% Here is text on Materials and Methods.
%
% \subsection{A descriptive heading about methods}
% More about Methods.
%
% \section{Data} (Or section title might be a descriptive heading about data)
%
% \section{Results} (Or section title might be a descriptive heading about the
% results)
%
% \section{Conclusions}



\section{Introduction}

Tropical cyclogenesis  typically proceeds from organized precipitating convection within deep saturated air columns  \cite<e.g.,>{E2018}. In principle, tropical cyclones can emerge spontaneously and no special precursors are necessary \cite<e.g,>{Hakim2011,W2020JAMES}.  That notwithstanding, in our current climate, tropical cyclones are  observed to form from mesoscale convection that is typically embedded within a preexisting larger-scale disturbance  \cite{SMA2012}. What elements of spontaneous self-aggregation are active within  preexisting synoptic-scale disturbances in a fully varying background flow? That question remains a subject of active research. Documenting, in detail, the characteristics of tropical cyclone precursors observed in nature is important in that regard. Here we examine some surface characteristics of African easterly waves (AEWs) during the time when they were developing into tropical cyclones. 


%How is this convection organized in the first place? There are several possibilities ---  for example, cloud-resolving numerical simulations have  convincingly shown that tropical convection can self-organize within an initially uniform environment in radiative-convective equilibrium \cite<e.g,>{Hakim2011,W2020JAMES}. In principle, tropical cyclones can emerge spontaneously and no special precursors are necessary.  That notwithstanding, in our current climate, tropical cyclones are  observed to form from mesoscale convection that is typically embedded within a preexisting larger-scale disturbance  \cite{SMA2012}.
%Such precursor disturbances are thought to help organize the convection while buffering it against deleterious effects of excessive vertical wind shear and low-entropy air intrusion \cite<e.g.,>{DMW09, TangEmanuel2010, TA2012}. What elements of spontaneous self-aggregation are active within a preexisting synoptic-scale disturbances in a fully varying background flow? That question remains a subject of active research. Documenting, in detail, the characteristics of tropical cyclone precursors observed in nature is important in that regard. Here we examine some surface characteristics of African easterly waves (AEWs) during the time when they were developing into tropical cyclones. 

%AEWs are synoptic scale disturbances connected to the African Monsoon \cite{C69,B72}. They are the precursors to nearly 60\% of Atlantic tropical cyclones \cite<e.g.,>{CWC2008,RAWH2017}. AEWs are, however, numerous and only a fraction of them transition into tropical cyclones \cite<e.g.,>{HTHA07}. The canonical structure of an AEW shows peak wave amplitude in the lower-to-mid troposphere \cite{KTH06,Russell_Aiyyer_2020}. Coherent peaks in vorticity within the waves can be tracked easily at both 850 and 700 hPa \cite{TH01}. These localized maxima in vorticity are thought to result from vortex stretching by the vertical mass flux in areas of deep moist convection.

%Something about surface vortex and TC genesis

%Numeroast studies have attempted to identify critical parameters of AEWs and other tropical waves that are associated with the formation of tropical cyclones. 


%Past studies have focused on a variety of ocean and atmospheric parameters that influence  tropical cyclogenesis from precursors such as easterly waves. 

\citeA{MZ81} found that developing easterly waves tended to have stronger low-level relative vorticity and weaker environmental vertical wind shear compared to non-developing ones. Subsequent studies have expanded the parameter space to include thermal structure, the vigor of precipitating convection, environmental moisture and convective cloud fraction \cite<e.g.,>{HTT2010,K2013,DAHM2014}.  \citeA{LCP2013} and \citeA{ZZ2014a} suggested that,  while the intensity of convection is not a discriminator of tropical cyclogenesis, developing easterly waves were associated with a greater fractional area of intense convection as compared to non-developing ones. \citeA{FWND2016GRL} and \citeA{Z2020MWR} reported enhanced  intensity and areal coverage of precipitation prior to tropical cyclogenesis. On the other hand, \citeA{WZ2018} reported large variability in the intensity, frequency, and area of deep convection during tropical cyclogenesis. Interestingly, she found one consistent feature---during tropical cyclogenesis, intense convection tended to cluster within the center of the incipient vortex while outside this core region, it remains unchanged or might even weaken.



%\citeA{FWND2016GRL} found enhanced precipitation intensity and areal coverage during 36 hours prior to tropical cyclogenesis, with increasing contribution from stratiform, mid-level and deep convection.  \citeA{Z2020MWR} found that the area covered by precipitation was higher in developing disturbances as compared to non-developing ones. 

The aforementioned studies have utilized a variety of data (e.g., global reanalysis, dropsondes, satellite-derived cloud properties, and precipitation), but have tended to focus on tropospheric parameters. Scant attention has been devoted to the role of surface enthalpy fluxes within the precursor waves prior to tropical cyclogenesis.  Indeed, \citeA{MB2018} noted that, in general, the role of surface enthalpy flux during the spin-up of a tropical cyclone is still being debated. On the other hand, once a tropical cyclone has formed, surface enthalpy fluxes have been shown to be critical for its subsequent intensification \cite{E2018}. One particular instability mechanism ---wind-induced surface heat exchange (WISHE)---relies on positive feedback between surface winds and enthalpy fluxes and is activated once a mesoscale saturated column of air is established \cite{ZE2016}. \citeA{MB2018} attempted to address the role of surface enthalpy fluxes during the initial spin-up of a tropical cyclone (i.e., the tropical depression stage) using idealized simulations. One of their key findings was that a negative radial gradient of surface enthalpy flux is necessary for the genesis of a tropical cyclone from a precursor vortex. 


The majority of past investigations of surface fluxes in tropical cyclones have relied on numerical simulations. Relatively few have been able to exploit direct flux observations \cite<e.g.,>{CBH2000M, BME2012}. As these measurements are typically sourced from buoys, field campaigns, and coastal observing stations, they lack the spatial and temporal coverage that is needed for detailed diagnostics. Some studies have used surface fluxes derived from remotely sensed data \cite<e.g.,>{LCCB2011}; but they have tended to focus on the intensification of tropical cyclones. To our knowledge, no prior study dealing with surface winds and enthalpy fluxes in  AEWs undergoing tropical cyclogenesis has been reported in the published literature. 

In this paper, we document the composite structure of surface winds and enthalpy (latent and sensible heat) fluxes associated with developing AEWs. We used data from the recently launched NASA Cyclone Global Navigation Satellite System (CYGNSS) mission which consists of a constellation of low-earth orbiting satellites \cite{2016BAMS}. 

\section{Data}

We used the following data covering July--October 2018-2021. 
\begin{itemize}
\item CYGNSS surface winds -- Level 3 Science Data Record (SDR), version 3.1 \cite{2016BAMS}. We use the fully developed seas (FDS) wind speeds that are provided hourly on a $0.2 \times 0.2^o$ grid within about 40$^o$ north and south of the equator. We averaged the hourly data to create daily mean fields prior to subsequent processing. 

\item 10m winds and sea level pressure from the ERA5 reanalysis \cite{era5}.

\item  CYGNSS surface latent and sensible heat flux (Level 2 SDR 2.0) that are based on the SDR 3.1 wind retrievals and ERA5 thermodynamic fields. Some additional information regarding the CYGNSS data, including an example of CYGNSS winds associated with a typical AEW,  is included in the supporting information. 
\item Following \citeA{RAWH2017}, to ascertain which Atlantic tropical cyclones developed from AEWs, we use the storm reports prepared by the US National hurricane center (NHC). We only considered those tropical cyclones that were specifically attributed to a wave that emerged from the west coast of Africa

%These reports provide a description of the precursor for each tropical cyclone. We only considered those tropical cyclones that were specifically attributed to a wave that emerged from the west coast of Africa. Tropical cyclogenesis (Day-0) was taken to be the day when a storm was first declared to be a tropical depression by NHC. 

%For subsequent calculations for leading and lagging days, this was defined as Day-0.
\end{itemize}


%The CYGNSS retrievals appear to capture the gross features of this wavetrain, with enhanced wind speed around the trough.



%{\bf To do: }Include info on validation of cygnss winds and fluxes; and limitations of the surface flux data. The use of bulk aerodynamic method using the COARE algorithm and how it is not valid beyond 25 m/s. 

\section{Results}


\subsection{Climatalogical Surface Winds Over The Tropical Atlantic}

We first show that the CYGNSS winds are capable of depicting the climatology mean as well as the synoptic variability of surface winds over the tropical Atlantic. The mean and variance of the daily averaged CYGNSS (FDS) winds for July--October 2018-2021, along with the  climatological (1980--2018) mean sea level pressure, is presented in Figure \ref{fig:climo}. The CYGNSS winds clearly show the presence of the Atlantic subtropical anticyclone, consistent with the spatial structure seen in the ERA5 sea level pressure contours. The low-level jet over the Caribbean can also be seen. This jet has been suggested to be important for the amplification of easterly waves crossing into the eastern Pacific \cite{MKDVS97}. The mean winds are generally weaker within the main development region MDR (marked by the rectangle). The low-level westerly monsoon flow can be deduced from the enhanced wind speeds over the near-equatorial eastern Atlantic. On the other hand, the African easterly jet (AEJ) which, on average is located around 12$^o$N and peaks in the mid-troposphere, does not appear to extend down to the surface as noted from the lack of any wind maximum off the coast of west Africa.

Figure \ref{fig:climo}b shows a zonally oriented region of enhanced wind variance within the MDR--- between 5$^o$N--15$^o$N, and from the west coast of Africa to 60$^o$W. This enhanced variance occurs where the mean wind is weak (Fig. \ref{fig:climo}a). The atmospheric variability in the off-equatorial tropical Atlantic is dominated by synoptic-scale waves during July--October \cite<e.g.,>{MTA06}. Thus, we infer that this region of enhanced variance depicts the surface signal of the AEW stormtrack in the CYGNSS winds. Albeit episodic, tropical cyclones will also contribute to the daily wind variance as discussed by \citeA{SMA2012}. Just off the west coast of Africa, around 20$^o$N, a small band of enhanced variance can be noted.  We associate this with the surface reflection of the northern AEW stormtrack that exists poleward of the African easterly jet \cite<e.g.,>{TP01, DA13a}. The northern AEW stormtrack appears to merge with the southern AEW stormtrack between 20$^o$W-30$^o$W. The aforementioned features seen in the variance of CYGNSS winds are consistent with AEW stormtracks seen in 850-hPa synoptic-scale eddy kinetic energy derived from global reanalysis fields \cite<e.g.>{Russell_Aiyyer_2020}. One additional feature is notable in Figure \ref{fig:climo} -- over the Caribbean, the enhanced surface wind variance is shifted west of the peak surface winds. This downstream shift of eddy activity relative to the low-level Caribbean jet is consistent with the notion that easterly waves may form or amplify owing to the instability of the background flow in this region \cite{MKDVS97}. 

\subsection{Composite Wind Structures During Tropical Cyclogenesis}
We now consider the evolution of surface winds during the time of tropical cyclogenesis from AEWs. A total of 31 tropical cyclones were identified by the NHC as originating from AEWs within the MDR during the study period. The genesis locations of these storms are shown in Figure S2. To document the surface wind evolution, we calculated  storm-relative composite means as follows. We shifted the data grids such that all storms shown in Figure S2 are co-located at a reference point (10$^o$N; 40$^o$W) on the day of tropical cyclogenesis (Day-0). For lag-composites, we moved the date of the composite forward and backward while retaining the same spatial shift. Although there may be considerable storm-to-storm variability, such compositing techniques elucidate the features that are most likely to occur in a synoptic phenomenon \cite{WZ2018}.

The composite CYGNSS (FDS) speeds and 10-m ERA5 velocity vectors are shown in the left panel of Figure \ref{fig:comp}. The right panel shows the same fields but as anomalies relative to a background mean that was calculated by averaging over 13 days centered on Day-0 for each storm in the composite.  For a vector in the composite field to be deemed statistically significant (shown by bold arrows), either its zonal or the meridional component must be significant.  The statistical significance of each wind component was evaluated by comparing it against 1000 composites, each created by randomly drawing 31 dates over July--October, 2018–2021. A two-tailed significance was evaluated at the 95\% confidence level with the null hypothesis being that the composite average could have resulted from a random draw. 

%For visualization, we show all vectors because, as seen in fig. \ref{fig:comp}, they represent a physically continuous and meaningful representation of the composite wave evolution.


Figure \ref{fig:comp} shows that there is a close correspondence between the structure of the CYGNSS retrievals and ERA5 near-surface winds. An incipient surface vortex (marked by the filled square) is beginning to appear on Day-4, and becomes more coherent on Day-3. The proto-vortex associated with the composite AEW moves westward and continues to amplify. In part, this increased coherence is expected simply as a result of the compositing method as we get closer to Day-0. Nevertheless, it shows that a surface-based vortex with closed circulation is in place 3--4 days prior to the tropical depression stage. This is consistent with the notion of a proto-vortex embedded within a synoptic scale wave pouch that is often visualized in a wave-following reference frame  \cite<e.g.,>{DMW09}. Interestingly, the composite surface vortex  is visible here even in the earth-relative frame.

 The leading and trailing anticyclonic anomalies straddling the main vortex can also be seen, particularly in the anomaly fields. Despite the minimal data filtering employed here--in the form of removing the 13-day mean flow---these features clearly highlight the AEW wavepacket in the surface wind fields, consistent with those seen in 2-10 day filtered fields at 850 hPa \cite<e.g.,>{DA13a}.

\section{Surface Enthalpy fluxes within AEWs}
 Figure \ref{fig:vplot_fds} illustrates the distribution of latent and sensible heat fluxes as a function of time relative to tropical cyclogenesis using data from all 31 AEWs considered in this study. For each day, we extracted all available flux values for each wave within a radial distance of 700 km from the center of the tracked composite vortex (see fig. \ref{fig:comp}). The extent of this region is roughly half the canonical AEW wavelength \cite<e.g.,>{DA15} and represents  the cyclonic circulation of the wave. The overall qualitative interpretation of our subsequent findings is not sensitive to the dimension of this bounding region as long as it encompasses the bulk of the AEW trough.

%Some numeric values for selected percentiles of the distributions are provided in Table S1.

Figure \ref{fig:vplot_fds}a shows an expansion of the upper extremes of the Latent heat flux distribution over time. The 99$^{th}$ and 95$^{th}$ percentile values increase by 36\% and 11\%, respectively, from Day-3 to Day +3. These increases occur subsequent to tropical cyclogenesis. On the other hand, the mean and median values barely change. They increase only by 5\% and 3\% respectively. On average, the sensible heat fluxes (\ref{fig:vplot_fds}b) are about 10 times smaller than the latent heat fluxes. From Day-3 to Day+3, the 99$^{th}$ and 95$^{th}$ percentile values of the sensible heat fluxes increase by 16\% and 8\%, respectively. Intriguingly, the mean and median of these fluxes decrease by roughly 10\% and 22\% respectively. From Fig. \ref{fig:vplot_fds}, it can be noted that this decrease occurs mostly after the tropical depression has formed. 

%Prior to that, the mean and median sensible heat fluxes remain relatively unchanged.

%=====================

The increase in the upper extremes of both sensible and latent fluxes is unsurprising since peak surface wind speeds increase after the genesis of the tropical cyclone. However, the key result from Fig. \ref{fig:vplot_fds} is that the mean surface enthalpy fluxes do not change substantially during the 3 days prior to tropical cyclogenesis. Rather, the bulk of the change occurs after the formation of the tropical depression, and likely reflects its subsequent intensification into a tropical cyclone.  Thus, the intensity of surface enthalpy fluxes within the AEW may not be a particularly good predictor of imminent cyclogenesis. On the other hand, the robust expansion of the upper extremes ($>90^{th}$ percentile) of their distributions suggests that localized sharp increases in surface fluxes accompany tropical cyclogenesis and further intensification. 


%The increase in the upper Furthermore, while the upper extremes do increase substantially,  the area-averaged intensity of the surface enthalpy fluxes, based on the metrics of mean and median of their distributions within the sub-synoptic region surrounding the developing vortex, do not dramatically increase even after tropical cyclogenesis. While these metrics for latent heat fluxes increase modestly, the same for sensible heat fluxes are lower after tropical cyclogenesis. 


%On the other hand, the robust reduction of mean and median sensible heat fluxes within the same sub-synoptic region the after the formation of the tropical depression needs additional scrutiny
%====================

We now examine whether there is a discernible change in the radial structure of the surface enthalpy fluxes in the developing vortex within the AEW. For each day relative to cyclogenesis, we binned all available fluxes (for all 31 AEWs) based on the distance from the center of the vortex tracked in Fig. \ref{fig:comp}. We show the results for 50 km bin width in Fig. \ref{fig:flux_rad_comp}. The interpretation was qualitatively similar when we used other reasonable values for the bin width ranging from 20--70 km. Fig. \ref{fig:flux_rad_comp} illustrates the result of the binning in two ways, representing simple measures of azimuthally averaged fluxes as a function of distance (radius) from the vortex center. The bars show the mean flux and its 95\% confidence interval for each bin, and the orange line shows the non-parametric locally weighted scatterplot smoothing (LOWESS) regression curve. The LOWESS curve was calculated using all data points prior to binning them.   

Two key observations from Fig. \ref{fig:flux_rad_comp} can be made. First, up to three days before the formation of the tropical depression, the sensible heat flux is nearly radially uniform. A similar picture was seen on  Day-4 and earlier (not shown). On the other hand, there is already a clear inward increase  (i.e., negative radial gradient) in the sensible heat flux by Day-3. Second, closer to cyclogenesis (Day-2 and Day-1), a negative radial gradient of latent heat fluxes is also evident. As expected, due to the way the flux data are aggregated based on the composite vortex center, the strength of this negative radial gradient is most striking on the reference day (Day 0). But despite the differences in the subsequent tracks and motion of the developing tropical cyclone, this negative radial gradient is also present on Day+1 and Day+2. \citeA{WZ2018} showed that, during the time leading up to cyclogenesis, intense convection appears to move towards the center of the proto-vortex. She also found that, in the outer parts of the proto-vortex, the intensity of convection is unchanged or even reduced. This concentration of convection is likely supported by increasing values of surface enthalpy fluxes in the core of the vortex as seen in Fig. \ref{fig:flux_rad_comp}, and is consistent with the modeling work of \citeA{MB2018}. 

Some features seen in Fig. \ref{fig:flux_rad_comp} need additional scrutiny. That the negative radial gradient is seen earlier for the sensible heat flux is an intriguing result. It also appears that the area-mean sensible heat flux is diminished by Day+2 as compared to earlier days. The reason for these observations is unclear from our analysis and calls for high-resolution numerical simulations with interactive air-sea coupling.
\section{Discussion}

%Our work has used a novel data set that is based on a blend of satellite wind retrievals and global reanalysis and provides additional evidence for the role of surface enthalpy fluxes and their radial distribution for tropical cyclogenesis.

%We examined the surface signature of AEWs that developed into tropical cyclones using CYGNSS wind and surface enthalpy products in combination with near-surface winds from ERA5. Data for a total of 31 AEWs during 2018--2021 were aggregated in the form of distributions and composite means. We found a clear surface signature of a cyclonic vortex within AEWs as early as 4 days prior to the depression stage of tropical cyclogenesis (Fig. \ref{fig:comp}). The existence of this proto-vortex is consistent with the notion that the large scale synoptic environment of the AEW provides an environment (pouch) for the development of a cyclonic circulation that is sustained by moist convection \cite{DMW09,Wang2012JAS, TA2012}. 


%Past studies have attempted to identify factors that influence the spin-up of a tropical cyclone from a convectively generated precursor vortex (such as the one associated with AEWs seen in Fig. \ref{fig:comp}). 


The establishment of a saturated column of air is a critical step toward cyclogenesis because evaporation-driven downdrafts are reduced and the environment becomes conducive for deep convection, setting the stage for the WISHE process \cite{E2018}. \citeA{MVC2004JAS} described two stages of hurricane development: A pre-WISHE stage, wherein the radial profile of near-surface equivalent potential temperature ($\theta_e$), a measure of moist entropy, was nearly radially uniform; and a WISHE stage with a single dominant surface vortex with a moist core and marked inward increase  (i.e., a negative gradient) of $\theta_e$, consistent with the steady-state model of \citeA{S2003QJRMS}. As noted by \citeA{MB2018}, axisymmetric models of tropical cyclogenesis require a negative gradient of column relative humidity \cite<e.g.,>{Emanuel1997JAS, Frisius2006, S2003QJRMS}. On the basis of their idealized numerical simulations, \citeA{MB2018} argued that an inward increase in surface enthalpy fluxes is one possible pathway to get a persistent saturated core within a precursor vortex.  This motivated us to examine  the evolution of surface latent heat fluxes within AEWs leading to tropical cyclogenesis. We summarize our results for two periods:

\begin{itemize}
    \item \emph{Prior to tropical depression formation}: The CYGNSS-derived surface latent heat fluxes within the composite precursor vortex increase only modestly during the period leading to the depression stage of tropical cyclogenesis (Fig. \ref{fig:vplot_fds}).  Since we did not compare developing and non-developing AEWs, we cannot ascertain whether these developing waves were associated with some minimal threshold of surface fluxes that would maintain the precursor vortex or AEW. However, it appears - at least from the aggregate view -- that a rapid increase in the average surface latent heat fluxes is not a necessary step for cyclogenesis; rather it occurs after the depression has formed.  Surface enthalpy fluxes are  nearly radially uniform up to 3 days prior to the formation of the composite tropical depression. Following  \citeA{MVC2004JAS}, this would correspond to the pre-WISHE stage of tropical cyclone development. 
    
    Interestingly, during the two days leading to cyclogenesis, a clear negative radial gradient of surface enthalpy fluxes is established. This suggests a spatial reorganization of surface fluxes -- and by extension, intense moist convection -- that favors a shift towards the core of the developing vortex. This is consistent with the findings of \citeA{WZ2018}.  This radially inward increase of surface enthalpy flux likely sets the stage for tropical cyclogenesis, and is consistent with the model simulations of tropical cyclone spin-up from a proto-vortex described by \citeA{MB2018}. 

    \item \emph{Post tropical depression formation}: There is a widening of the surface enthalpy flux distribution towards higher values within the developing vortex (Fig. \ref{fig:vplot_fds}). The upper extreme of the latent heat flux distribution increases substantially a day after the depression has formed. As noted by \cite{WZ2018}, the  migration of convection towards the core of the developing vortex is the key feature of tropical cyclogenesis. Importantly, and related to the  migration of deep convection, a clear inward increase of surface enthalpy fluxes becomes a persistent feature (Fig. \ref{fig:flux_rad_comp}).  Following the arguments of  \citeA{MB2018} and \citeA{MVC2004JAS}, the rapid increase in the latent heat fluxes and the presence of a negative radial gradient of the fluxes in the  intensifying vortex indicates that the  WISHE mechanism is now fully active.
\end{itemize}

There are a few caveats to consider in relation to the data and method used here. The surface enthalpy fluxes are dependent on the fidelity of ERA5's surface thermodynamic fields and incur the attendant errors and biases. Additionally, we did not track the precursor AEWs, and instead used lag-composites to visualize the evolution of the fields. An alternative method would be to track the waves, which introduces other uncertainties stemming from multiple vorticity centers, splits, and mergers. We favor our current method 
for the ease of reproducibility. The coherence of the composite fields at different times (Fig. \ref{fig:comp}) gives us confidence that the wave-to-wave variability in tracks does not alter our conclusions.

%Additionally, the CYGNSS data has large spatial and temporal gaps owing to nature of the measurement swath. The use of composite averages mitigates some the deficiencies of the above two points. 

%Furthermore, t
%However, tracking AEWs with multiple vorticity peaks has its own limitations and uncertainties.


\section{Conclusions}

The CYGNSS retrievals capture the mean spatial structure of surface winds over the tropical Atlantic and the synoptic-scale AEW stormtrack. Lag-composites of CYGNSS and ERA5 data show a clear signal of an AEW wavepacket and an attendant, low-level cyclonic vortex as early as 3 days prior to the tropical cyclogenesis. The distribution of surface enthalpy fluxes within the proto-vortex does not change substantially prior to the tropical depression formation. Subsequently, the distribution widens, indicating amplified fluxes. Up to 3 days before the  depression stage, the surface latent fluxes within the precursor vortex are nearly radially uniform. Subsequently, from Day-2 onward, a clear negative radial gradient of both latent and sensible heat fluxes is established. These results are consistent with a recent modeling study that found that an inward increase of surface enthalpy fluxes is important for the spin-up of tropical cyclones \cite{MB2018}.

%and indicates the activation of the WISHE mechanism for tropical cyclone development \cite{MVC2004JAS}. 


%On the synoptic scale, AEW wavepackets can be seen as alternating cyclonic and anticyclonic circulations even in this minimally filtered analysis.
%


%Text here ===>>>


%%

%  Numbered lines in equations:
%  To add line numbers to lines in equations,
%  \begin{linenomath*}
%  \begin{equation}
%  \end{equation}
%  \end{linenomath*}



%% Enter Figures and Tables near as possible to where they are first mentioned:
%
% DO NOT USE \psfrag or \subfigure commands.
%
% Figure captions go below the figure.
% Table titles go above tables;  other caption information
%  should be placed in last line of the table, using
% \multicolumn2l{$^a$ This is a table note.}
%
%----------------
% EXAMPLE FIGURES
%
% \begin{figure}
% \includegraphics{example.png}
% \caption{caption}
% \end{figure}
%
% Giving latex a width will help it to scale the figure properly. A simple trick is to use \textwidth. Try this if large figures run off the side of the page.
% \begin{figure}
% \noindent\includegraphics[width=\textwidth]{anothersample.png}
%\caption{caption}
%\label{pngfiguresample}
%\end{figure}
%
%
% If you get an error about an unknown bounding box, try specifying the width and height of the figure with the natwidth and natheight options. This is common when trying to add a PDF figure without pdflatex.
% \begin{figure}
% \noindent\includegraphics[natwidth=800px,natheight=600px]{samplefigure.pdf}
%\caption{caption}
%\label{pdffiguresample}
%\end{figure}
%
%
% PDFLatex does not seem to be able to process EPS figures. You may want to try the epstopdf package.
%

%
% ---------------
% EXAMPLE TABLE
%
% \begin{table}
% \caption{Time of the Transition Between Phase 1 and Phase 2$^{a}$}
% \centering
% \begin{tabular}{l c}
% \hline
%  Run  & Time (min)  \\
% \hline
%   $l1$  & 260   \\
%   $l2$  & 300   \\
%   $l3$  & 340   \\
%   $h1$  & 270   \\
%   $h2$  & 250   \\
%   $h3$  & 380   \\
%   $r1$  & 370   \\
%   $r2$  & 390   \\
% \hline
% \multicolumn{2}{l}{$^{a}$Footnote text here.}
% \end{tabular}
% \end{table}

%% SIDEWAYS FIGURE and TABLE
% AGU prefers the use of {sidewaystable} over {landscapetable} as it causes fewer problems.
%
%\begin{sidewaysfigure}
% \includegraphics[width=20pc]{figs/climo.png}
% \caption{caption here}
% \label{newfig}
%\end{sidewaysfigure}

\section{Open Research}
The CYGNSS wind and surface enthalpy fluxes are available at, respectively, \url{https://doi.org/10.5067/CYGNS-L3X31} and  \url{https://doi.org/10.5067/CYGNS-L2H20}. The ERA5 fields are available at  \url{https://cds.climate.copernicus.eu}. The IBTRACS database of tropical cyclone tracks is archived at \url{https://doi.org/10.25921/82ty-9e16}.


\acknowledgments
We are grateful to Juan Crespo for the CYGNSS heat fluxes. This work was supported by NASA through awards NNX17AH61G and 80NSSC22K0610.


\newpage


\begin{figure}
 \includegraphics[width=.97\textwidth]{climo.png}
 \caption{Shaded fields showing (a) Variance ($m^{2}s^{-2}$) ; and (b) mean ($ms^{-1}$)  of daily averaged CYGNSS FDS wind speed over July-October 2018--2021. The contours on both panels show the long-term (1980-2018) climatology of mean sea level pressure from ERA5. The magenta rectangle marks the main development region considered in this paper.}
 \label{fig:climo}
\end{figure}

%
\begin{figure}
\centering{
 \includegraphics[width=25pc]{ERA_L3_comp.png}
 }
 \caption{Storm centered lag composite mean (left panels) and anomalies (Right panels) of CYGNSS FDS wind speeds (shaded) and ERA5 10m wind vectors. The star symbol marks the center of the composite storm at the first recorded depression stage corresponding to Day 0. The black square shows the incipient vortex within the composite AEW, and the hurricane symbol marks the location of the composite tropical storm after it has formed.}
 \label{fig:comp}
\end{figure}


\begin{figure}
 \includegraphics[width=.97\textwidth]{v_plot_fds.png}
 \caption{Distribution of (a) latent; and (b) sensible heat fluxes (Wm$^{-2}$) within a radial distance of 700 km from the composite vortex center (see Fig. \ref{fig:comp}) at different days relative to tropical cyclogenesis. The blue solid line marks the mean of the distributions. The rest of the solid lines depict, from bottom to top, the following percentiles of the distribution: P$_5$, P$_{10}$,P$_{50}$ (median), P$_{90}$, P$_{95}$, and P$_{99}$.}
 \label{fig:vplot_fds}
\end{figure}


\begin{figure}
 \includegraphics[width=.97\textwidth]{fluxes_radius_fds.png}
 \caption{Latent (top panels) and sensible (bottom panels) heat fluxes (Wm$^{-2}$) as a function of distance (km) from the vortex center at different days relative to tropical cyclogenesis. The fluxes were binned at 50 km radial increments. The bars show the  mean flux for each bin and its 95\% confidence interval. The orange line shows the non parametric LOWESS regression fit that was calculated using un-binned data}
 \label{fig:flux_rad_comp}
\end{figure}



%  \begin{sidewaystable}
%  \caption{Caption here}
% \label{tab:signif_gap_clos}
%  \begin{tabular}{ccc}
% one&two&three\\
% four&five&six
%  \end{tabular}
%  \end{sidewaystable}

%% If using numbered lines, please surround equations with \begin{linenomath*}...\end{linenomath*}
%\begin{linenomath*}
%\begin{equation}
%y|{f} \sim g(m, \sigma),
%\end{equation}
%\end{linenomath*}

%%% End of body of article

%%%%%%%%%%%%%%%%%%%%%%%%%%%%%%%%
%% Optional Appendix goes here
%
% The \appendix command resets counters and redefines section heads
%
% After typing \appendix
%
%\section{Here Is Appendix Title}
% will show
% A: Here Is Appendix Title
%
%\appendix
%\section{Here is a sample appendix}

%%%%%%%%%%%%%%%%%%%%%%%%%%%%%%%%%%%%%%%%%%%%%%%%%%%%%%%%%%%%%%%%
%
% Optional Glossary, Notation or Acronym section goes here:
%
%%%%%%%%%%%%%%
% Glossary is only allowed in Reviews of Geophysics
%  \begin{glossary}
%  \term{Term}
%   Term Definition here
%  \term{Term}
%   Term Definition here
%  \term{Term}
%   Term Definition here
%  \end{glossary}

%
%%%%%%%%%%%%%%
% Acronyms
%   \begin{acronyms}
%   \acro{Acronym}
%   Definition here
%   \acro{EMOS}
%   Ensemble model output statistics
%   \acro{ECMWF}
%   Centre for Medium-Range Weather Forecasts
%   \end{acronyms}

%
%%%%%%%%%%%%%%
% Notation
%   \begin{notation}
%   \notation{$a+b$} Notation Definition here
%   \notation{$e=mc^2$}
%   Equation in German-born physicist Albert Einstein's theory of special
%  relativity that showed that the increased relativistic mass ($m$) of a
%  body comes from the energy of motion of the body—that is, its kinetic
%  energy ($E$)—divided by the speed of light squared ($c^2$).
%   \end{notation}




%%%%%%%%%%%%%%%%%%%%%%%%%%%%%%%%%%%%%%%%%%%%%%%%%%%%%%%%%%%%%%%%
%
%  ACKNOWLEDGMENTS
%
% The acknowledgments must list:
%
% >>>>	A statement that indicates to the reader where the data
% 	supporting the conclusions can be obtained (for example, in the
% 	references, tables, supporting information, and other databases).
%
% 	All funding sources related to this work from all authors
%
% 	Any real or perceived financial conflicts of interests for any
%	author
%
% 	Other affiliations for any author that may be perceived as
% 	having a conflict of interest with respect to the results of this
% 	paper.
%
%
% It is also the appropriate place to thank colleagues and other contributors.
% AGU does not normally allow dedications.


%% ------------------------------------------------------------------------ %%
%% References and Citations

%%%%%%%%%%%%%%%%%%%%%%%%%%%%%%%%%%%%%%%%%%%%%%%
%
% \bibliography{<name of your .bib file>} don't specify the file extension
%
% don't specify bibliographystyle
%%%%%%%%%%%%%%%%%%%%%%%%%%%%%%%%%%%%%%%%%%%%%%%

\clearpage
\bibliography{aiyyer_references}




%Reference citation instructions and examples:
%
% Please use ONLY \cite and \citeA for reference citations.
% \cite for parenthetical references
% ...as shown in recent studies (Simpson et al., 2019)
% \citeA for in-text citations
% ...Simpson et al. (2019) have shown...
%
%
%...as shown by \citeA{jskilby}.
%...as shown by \citeA{lewin76}, \citeA{carson86}, \citeA{bartoldy02}, and \citeA{rinaldi03}.
%...has been shown \cite{jskilbye}.
%...has been shown \cite{lewin76,carson86,bartoldy02,rinaldi03}.
%... \cite <i.e.>[]{lewin76,carson86,bartoldy02,rinaldi03}.
%...has been shown by \cite <e.g.,>[and others]{lewin76}.
%
% apacite uses < > for prenotes and [ ] for postnotes
% DO NOT use other cite commands (e.g., \citet, \citep, \citeyear, \nocite, \citealp, etc.).
%



\end{document}


